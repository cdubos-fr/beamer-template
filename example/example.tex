\documentclass[aspectratio=169, xetex, 12pt]{beamer}
\usepackage[utf8]{inputenc}
\usepackage[T1]{fontenc}
\usepackage{datetime}

\usetheme{cdubos}

\title{Example de titre \\Sur plusieur ligne}

\begin{document}

    \begin{frame}
        \titlepage
    \end{frame}

    \begin{frame}{Avant de commencer}{Présention Python}
        \begin{block}{Python, c'est quoi}
            \begin{minipage}{0.7\paperwidth}
                \begin{itemize}
                    \item python c'est un langage de programation
                    \item 1 standards
                    \item Plusieurs implémentation
                \end{itemize}
            \end{minipage}
        \end{block}
        \begin{block}{Pourquoi faire}
            \begin{minipage}{0.7\paperwidth}
                \begin{itemize}
                    \item De la data
                    \item Du Web
                    \item Des logiciels
                \end{itemize}
            \end{minipage}
            \begin{minipage}{0.2\paperwidth}
                \begin{center}
                    \includegraphics[height=1cm]{media/python.png}
                \end{center}
            \end{minipage}
        \end{block}
    \end{frame}

    \section{hello}
    \subsection{Hello2}
    \begin{frame}
    \frametitle{A title}
    \framesubtitle{The proof uses \textit{reductio ad absurdum}.}
    \begin{theorem}
    There is no largest prime number. \end{theorem}
    \begin{enumerate}
    \item<1-| alert@1> Suppose $p$ were the largest prime number.
    \item<2-> Let $q$ be the product of the first $p$ numbers.
    \item<3-> Then $q+1$ is not divisible by any of them.
    \item<1-> But $q + 1$ is greater than $1$, thus divisible by some prime
    number not in the first $p$ numbers.
    \end{enumerate}
    \end{frame}

\end{document}
